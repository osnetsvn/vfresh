\begin{abstract}
Hypervisors are increasingly complex and  must be often updated for 
applying security patches, bug fixes, and feature upgrades. 
However, in a virtualized cloud infrastructure, 
updates to an operational hypervisor can be highly disruptive. 
Before being updated, virtual machines (VMs) 
running on a hypervisor must be either migrated away or shut down,
resulting in downtime, performance loss, and network overhead.
%Live migration of guest between 
%hosts also introduces significant network traffic and requires spare hardware 
%resources. 
We present a new technique, called \arch, to transparently replace 
a hypervisor with a new updated instance without disrupting 
any running VMs.
% technique to perform live and transparent replacement of a hypervisor with 
%minimal disruption to guests. 
A thin shim layer, called the hyperplexor, performs live hypervisor replacement by
remapping guest memory to a new updated hypervisor on the same machine.
The hyperplexor leverages nested virtualization for
hypervisor replacement while minimizing nesting overheads 
during normal execution.
We present a prototype implementation of the hyperplexor on 
the KVM/QEMU platform that can perform live hypervisor replacement within 10ms.
We also demonstrate how our hyperplexor-based approach can be used 
with CRIU-based process/container migration to perform
sub-second replacement of the underlying operating system.
\end{abstract}
